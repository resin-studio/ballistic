\documentclass[runningheads]{llncs}

% \documentclass{article}

\usepackage[
backend=biber,
style=alphabetic,
]{biblatex}

\usepackage{mathpartir}
\usepackage{hyperref}
\usepackage{mathtools}
\usepackage{stmaryrd}
\usepackage{listings}

\makeatletter % allow us to mention @-commands
\def\arcr{\@arraycr}
\makeatother

\lstset{
    % identifierstyle=\color{violet},
    % textcolor=blue,
    % keywordstyle=\color{blue},
    keywordstyle=\text,
    basicstyle=\ttfamily,
    mathescape=true,
    showspaces=false,
    morekeywords={let, fix, in}
}
\usepackage[utf8]{inputenc}
% \usepackage[T1]{fontenc}


\title{Progress report: Guiding synthesis of stochastic programs with probabilistic and statistical types}
\author{Thomas Logan}
\institute{University of Texas at Austin}

\begin{document}

\maketitle

\section{Initial progress}
I have designed a very simple language for constructing distributions and learning posterior distributions for 
latent variables/parameters from data. Learning from data is achieved by translating a simple program in my DSL into Pyro models and running Pyro's learning algorithms (stochastic variational inference).
\newline

\noindent
The following program in my DSL represents a template that generates distributions modulo a dataset.
\begin{lstlisting}
is_cont_africa, ruggedness => 
{log_gdp | is_cont_africa, ruggedness, log_gdp : data}
    a ~ normal(0., 10.) ; 
    b_a ~ normal(0., 1.) ; 
    b_r ~ normal(0., 1.) ; 
    b_ar ~ normal(0., 1.) ;
    sigma ~ uniform(0., 10.) ;
    mean ~ direct(
        a + 
        b_a * is_cont_africa + 
        b_r * ruggedness + 
        b_ar * is_cont_africa * ruggedness);
    normal(mean, sigma)
\end{lstlisting}

\noindent
The dataset is provided as input at the call site to compile/learn the program.
\begin{lstlisting}[language=Python]
DATA_URL = "https://d2hg8soec8ck9v.cloudfront.net/datasets/rugged_data.csv"
data = pd.read_csv(DATA_URL, encoding="ISO-8859-1")
df = data[["cont_africa", "rugged", "rgdppc_2000"]]

df = df[np.isfinite(df.rgdppc_2000)]
df["rgdppc_2000"] = np.log(df["rgdppc_2000"])

data = torch.tensor(df.values, dtype=torch.float)

result = generate_function(util.resource('examples/program.bll'), data)
\end{lstlisting}

\noindent
Generating the above program results in constructing the following Pyro code.
\begin{lstlisting}[language=Python]
def model(is_cont_africa, ruggedness, obs=None):
    a = pyro.sample("a", dist.Normal(0.0, 10.0))
    b_a = pyro.sample("b_a", dist.Normal(0.0, 1.0))
    b_r = pyro.sample("b_r", dist.Normal(0.0, 1.0))
    b_ar = pyro.sample("b_ar", dist.Normal(0.0, 1.0))
    sigma = pyro.sample("sigma", dist.Uniform(0.0, 10.0))
    mean = 
        a+
        b_a*is_cont_africa+
        b_r*ruggedness+
        b_ar*is_cont_africa*ruggedness
    
    with pyro.plate("data", len(is_cont_africa)):
        return pyro.sample("obs", 
            dist.Normal(mean, sigma), obs=obs)

auto_guide = pyro.infer.autoguide.AutoNormal(model)
adam = pyro.optim.Adam({"lr": 0.02}) 
elbo = pyro.infer.Trace_ELBO()
svi = pyro.infer.SVI(model, auto_guide, adam, elbo)

losses = []
for step in range(1000 if not smoke_test else 2): 
    loss = svi.step(data[:, 0], data[:, 1], data[:, 2])
    losses.append(loss)
        
predictive = pyro.infer.Predictive(model, 
    guide=auto_guide, num_samples=100)

def multi(is_cont_africa, ruggedness):
    global predictive 
    svi_samples = predictive(is_cont_africa, ruggedness)
    svi_gdp = svi_samples["obs"]
    return svi_gdp

def single(is_cont_africa, ruggedness):
    global predictive 
    svi_samples = predictive(
        torch.tensor([is_cont_africa]), 
        torch.tensor([ruggedness])
    )
    svi_gdp = svi_samples["obs"]
    return svi_gdp[:, 0]
\end{lstlisting}

\noindent
Comparing the source code to the generated Pyro code illustrates the improved usability of the DSL. 

\section{Next steps}
The next steps are to enable automatic discovery of the program architecture. 
I'm considering using tree-automata to search for programs. 
Another issue is how to partially specify a subset of the DSL to explore. 
Perhaps specifications could look like predicate constraints, but we would need to distinguish between
constraints for learning parameters (the data) and constraints for specifying architecture. 

\end{document}
